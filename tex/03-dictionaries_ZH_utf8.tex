\chapter{词典}
\label{chap-dictionaries}

\section{DELA词典}
\index{DELA}\index{Dictionnaire!format}\index{LADL}

Unitex 所使用的电子词典使用\ DELA 形式(LADL 电子词典)。这种形式能描述用户输入的某种语言的简单或复合的词目,\index{Entrée lexicale}语法上的任意搭配,词义及词形变化。电子词典共有两种类型。第一种更加常用的带有变位形式的字典,叫做\ DELAF(DELA 的变位形式)或\ DELACF,它是针对复合词的字典。第二类型的字典叫做\ DELAS(DELA简单形式)或\ DELAC。Unitex  的项目并不区分简单形式和复合形式的字典。我们使用\ DELAF 和\ DELAS 两种词典,来表示其中词条为单词、词组的或复合词的的词目。 

\subsection{DELAF 的形式}
\label{section-DELAF-format}
\subsubsection{输入的语法}
\label{section-DELAF-entry-syntax}
一个\ DELAF 的输入就是一行以回车结束的文本,它符合以下的规则:

\bigskip
\begin{verbatim}
mercantiles,mercantile.A+z1:mp:fp/ceci est un exemple
\end{verbatim}

\bigskip
\noindent 组成这一行的元素如下:

\bigskip
\begin{itemize}
\item \verb+mercantiles + 是这条词目的变化形式。\index{Forme!fléchie} 这个变化形式是必须的;
  
\bigskip \item \verb+mercantile+ 是这条词目的标准形式。
\index{Lemme}\index{Forme!canonique} 对于名词和形容词,它一般是阳性单数形式;对于动词,标准形式指该动词的原型。这条信息可以被省略,就像以下这个例子:

  
\bigskip
\verb$boîte à merveilles,.N+z1:fs$
  
\bigskip 该词目的标准形式正是它的变化形式。标准形式和变化形式之间用逗号隔开;
\index{\verb+,+}
  
\bigskip \item \verb$A+z1$ 是该词目的语法和词义信息。,
\index{Informations!grammaticales}\index{Informations!sémantiques} 在我们的例子中\ \verb+A+
表示形容词, \verb+z1+ 代表当前词 (见表\ ~\ref{tab-semantic-codes}).

每一条词目都至少有一条语法或语义代码,用句号和标准形式隔开。如果有许多条信息,互相之间用加号隔开。 \verb$+$\index{\verb$+$}\index{\verb+.+}.
  
\bigskip
\item \verb+:mp:fp+ 是词形变化码。
\index{Informations!flexionnelles} 它描述了该词目的种类,数量,时态和语态,变化形式等。这些信息不是必须的,一条词目可包含一条或多条变化信息,互相之间用冒号隔开。在我们的例子中, \verb+m+ 表示阳性,
\verb+p+ 表示复数 \verb+f+ 表示阴性 (见表\ ~\ref{tab-inflectional-codes}). 字符\ \verb+:+ 可理解为“或”。 如\ \verb+:mp:fp+ 代表“阳性复数”或“阴性复数”。 C 因为每一个字母代表一条信息,没有必要重复使用相同的字母。如:用\ \verb+:PP+ 表示过去分词完全等于用\ \verb+:P+ 单独表示;\index{\verb+:+}
  
\bigskip \item \verb+/ 是一条注释。注释不是必须的,它可以用\ \verb+/+ 来表示。当人们精简字典时注释将被删除。\index{Commentaire!dans un dictionnaire} \index{Dictionnaire!commentaire}\index{\verb+/+}
\end{itemize}

\bigskip
\noindent 重要提示:逗号和句号可以在字典中被使用。为了完成这一点,需要同时使用以下转义符:
\verb+\+ \index{\verb+\,+}\index{\verb+\.+}\index{\verb+\+}:

\bigskip
\begin{verbatim}
3\,1415,PI.NOMBRE
Organisation des Nations Unies,O\.N\.U\..SIGLE
\end{verbatim}


\bigskip
\noindent 注意:每个字符包含在一个字典条目中。例如,如果输入的空间中,这些将被认为的信息的一个组成部分。在下面一行:


\begin{verbatim}
gît,gésir.V+z1:P3s /voir ci-gît
\end{verbatim}

\bigskip \noindent 字符前的空格\ \verb+/+ 被看做词形变化码,词形变化码包含四个字符\ \verb+P+, \verb+3+, \verb+s+ 和一个空格符。


\bigskip \noindent 可以在词典\ DELAF 或\ DELAS 中输入
,用\ $/$来 表示行。例如,

\bigskip
\begin{verbatim}
/ L’entrée nominale pour ’par’ est un terme de golf
par,.N+z3:ms
\end{verbatim}


\subsubsection{有空格或破折号的词}

\index{Mots!composés!avec espace ou tiret}\index{\verb+=+}\index{\verb+\=+}

某些复合词如\ \textit{grand-mère} 可以用空格或破折号来写。为了避免重复的条目,可以使用\\verb+=+ . 在当字典被压缩时,程序\ \verb+Compress+
\index{Programmes externes!\verb+Compress+}\index{\verb+Compress+} 检查每一行。
如果在词尾变化形式或规范形式包含了非转义字符\ \verb+=+。 如果是这种情况,该程序由两个条目替换本:一种其中\ = 字符由一个空取代,和一个在那里由破折号取代。因此,下面的条目:


\bigskip \verb$grand=mères,grand=mère.N:fp$

\bigskip
\noindent 被以下两行替代

\bigskip
\verb$grand mères,grand mère.N:fp$

\verb$grand-mères,grand-mère.N:fp$


\bigskip
\noindent 注:如果您想保留包含\ \verb+=+ 字符的条目,使用\ \verb+\+ ,如下面的例子:


\bigskip
\verb$E\=mc2,.FORMULE$\\

当字典被压缩,本次置换完成。在压缩字典中,转义字符\ \verb+=+ 被简单 的\ \verb+=+ 替换。
因此,如果以下行被压缩: 

\begin{verbatim}
E=mc2,.FORMULE
grand=mère,.N:fs
\end{verbatim}

\noindent 我们将字典运用到以下文本:

\verb$Ma grand-mère m’a expliqué la formule E=mc2.$


\bigskip \noindent 以下行会在文本的合成词词典获得:


\begin{verbatim}
E=mc2,.FORMULE
grand-mère,.N:fs
\end{verbatim}


\subsubsection{词条分解}

具有相同的词形变化和规范化形式的多个条目可以被组合成一种情况,只要它具有相同的语法和语义码。:


\bigskip
\begin{verbatim}
glace,glacer.V+z1:P1s:P3s:S1s:S3s:Y2s
\end{verbatim}

\bigskip 
\noindent 如果语法和语义信息不同,需要分别创建条目:


\bigskip
\begin{verbatim}
glace,.N+z1:fs
glace,glacer.V+z1:P1s:P3s:S1s:S3s:Y2s
\end{verbatim}

\bigskip 
\noindent 

一些有相同的语法和语义码的词条可以有不同的含义,比如单词\ \textit{poêle},阳性可以指加热器或面纱,阴性可以指厨房用具。因此我们可以根据以下情况进行区分:



\bigskip
\noindent
\texttt{poêle,.N+z1:fs/ poêle à frire}

\noindent
\texttt{poêle,.N+z1:ms/ voile, linceul; appareil de chauffage}

\bigskip 
\noindent 注意:在实践中,这种区分仅仅是字典输入增加的结果,如果我们根据下列方式合并输入条目,Unitex 的各个程序也会得到相同的结果。
\bigskip
\noindent
\texttt{poêle,.N+z1:fs:ms}

\bigskip 
\noindent 因此这种情况下,是否要分别创建条目取决于字典创建者的偏好。


\index{DELAF|)}\index{Dictionnaire!DELAF|)}

\subsection{DELAS format}
\label{section-DELAS-format}
\index{DELAS}\index{Dictionnaire!DELAS}

DELAS的格式非常类似于\ DELAF 的。不同的是,只提到了规范的形式后的语法码或语义码。规范形式是用逗号分隔不同的代码。下面是一个例子:


\begin{verbatim}
cheval,N4+Anl
\end{verbatim}

\noindent 
词形变化程序会将条目的第一个语法码或语义码翻译成语法名,用于输入进词形变化程序。下面的例子表示\ \textit{cheval} 需要使用\ \verb+N4+ 命名程序来进行词形转换。本可以在输入时加入词形变化码,但是词形变化程序的操作属性限制了这么做的可能性。有关详细信息,请参阅下文本章节 ~\ref{section-automatic-inflection}.


\subsection{内容框}
\index{Dictionnaire contenu}\index{Dictionnaire codes utilises}

提供\ Unitex 的词典包含了简单复合词的描述。这些描述显示每个条目的语法目录,可能的词形变化码和各种语义信息。下列表格提供用\ Unitex 提供使用的字典中的不同代码的概述。这些代码有几乎所有的语言相同的含义,甚至其中有一些适用于特定的语言\  (\textit{i.e.} marque du neutre, 等。).

\begin{table}[!h]
\index{\verb+A+}\index{\verb+ADV+}\index{\verb+CONJC+}\index{\verb+CONJS+}\index{\verb+DET+}
\index{\verb+INTJ+}\index{\verb+N+}\index{\verb+PREP+}\index{\verb+PRO+}\index{\verb+V+}
\begin{center}
\begin{tabular}{|c|l|l|}
\hline
\textbf{Code} & \textbf{Signification} & \textbf{Exemples} \\
\hline
\verb+A+ & adjectif & fabuleux, broken-down \\
\hline
\verb+ADV+ & adverbe & réellement, à la longue \\
\hline
\verb+CONJC+ & conjonction de coordination & mais\\
\hline
\verb+CONJS+ & conjonction de subordination & puisque, à moins que \\
\hline
\verb+DET+ & déterminant & ses, trente-six \\
\hline
\verb+INTJ+ & interjection & adieu, mille millions de mille sabords \\
\hline
\verb+N+ & nom & prairie, vie sociale\\
\hline
\verb+PREP+ & préposition & sans, à la lumière de \\
\hline
\verb+PRO+ & pronom & tu, elle-même \\
\hline
\verb+V+ & verbe & continuer, copier-coller\\
\hline
\end{tabular}
\caption{Codes grammaticaux usuels\label{tab-grammatical-codes}}
\end{center}
\end{table}
\vspace{-0.7cm}
\begin{table}[!h]
\index{\verb+z1+}\index{\verb+z2+}\index{\verb+z3+}\index{\verb+Abst+}\index{\verb+Anl+}\index{\verb+AnlColl+}
\index{\verb+Conc+}\index{\verb+ConcColl+}\index{\verb+Hum+}\index{\verb+HumColl+}\index{\verb+t+}\index{\verb+i+}
\index{\verb+en+}\index{\verb+se+}\index{\verb+ne+}
\begin{center}
\begin{tabular}{|c|l|l|}
\hline
\textbf{Code} & \textbf{Signification} & \textbf{Exemple} \\
\hline
\verb+z1+ & langage courant & blague \\
\hline
\verb+z2+ & langage spécialisé & sépulcre \\
\hline
\verb+z3+ & langage très spécialisé & houer \\
\hline
\verb+Abst+ & abstrait & bon goût \\
\hline
\verb+Anl+ & animal & cheval de race \\
\hline
\verb+AnlColl+ & animal collectif & troupeau \\
\hline
\verb+Conc+ & concret & abbaye \\
\hline
\verb+ConcColl+ & concret collectif & décombres \\
\hline
\verb+Hum+ & humain & diplomate \\
\hline
\verb+HumColl+ & humain collectif & vieille garde \\
\hline
\verb+t+ & verbe transitif & foudroyer \\
\hline
\verb+i+ & verbe intransitif & fraterniser \\
\hline
\verb+en+ & particule pré-verbale (PPV) obligatoire & en imposer \\
\hline
\verb+se+ & verbe pronominal & se marier \\
\hline
\verb+ne+ & verbe à négation obligatoire & ne pas cesser de \\
\hline
\end{tabular}
\caption{Quelques codes sémantiques\label{tab-semantic-codes}}
\end{center}
\end{table}

%\bigskip
\noindent 注:计时表的说明\ ~\ref{tab-inflectiona-codes} 对应的是法文。然而,大多数定义都可以在不同的词性(不定式,现在分词等)中找到。


\bigskip
\noindent 因为大多数语言的具有共同基础,字典包含特定编码具体到每一种语言。因为词形变化码在不同语言中有很大的不同,这里不做详细介绍。对于在字典中使用的所有代码的完整说明,我们建议您直接联系字典的作者。


\begin{table}[!h]
\index{\verb+m+}\index{\verb+f+}\index{\verb+n+}\index{\verb+s+}\index{\verb+p+}\index{\verb+1+}
\index{\verb+2+}\index{\verb+3+}\index{\verb+P+}\index{\verb+I+}\index{\verb+S+}\index{\verb+T+}
\index{\verb+Y+}\index{\verb+C+}\index{\verb+J+}\index{\verb+W+}\index{\verb+G+}\index{\verb+K+}
\index{\verb+F+}
\begin{center}
\begin{tabular}{|c|l|}
\hline
\textbf{Code} & \textbf{Signification} \\
\hline
\verb+m+ & masculin \\
\hline
\verb+f+ & féminin \\
\hline
\verb+n+ & neutre \\
\hline
\verb+s+ & singulier \\
\hline
\verb+p+ & pluriel \\
\hline
\verb+1+, \verb+2+, \verb+3+ & 1st, 2nd, 3rd personne\\
\hline
\verb+P+ & présent de l’indicatif \\
\hline
\verb+I+ & imparfait de l’indicatif  \\
\hline
\verb+S+ & présent du subjonctif\\
\hline
\verb+T+ & imparfait du subjonctif \\
\hline
\verb+Y+ & présent de l’impératif \\
\hline
\verb+C+ & présent du conditionnel\\
\hline
\verb+J+ & passé simple \\
\hline
\verb+W+ & infinitif \\
\hline
\verb+G+ & participe présent \\
\hline
\verb+K+ & participe passé \\
\hline
\verb+F+ & futur \\
\hline
\end{tabular}
\caption{Codes flexionnels usuels\label{tab-inflectional-codes}}
\end{center}
\end{table}


\bigskip
\noindent 显示的代码是绝对不限制。每个用户都可以将自己的代码,并创建自己的字典。例如,出于教育性质的目的,我们可以引入一些标签来表示法语中的\ faux-amis。

\bigskip
\begin{verbatim}
bless,.V+faux-ami/bénir
cask,.N+faux-ami/tonneau
journey,.N+faux-ami/voyage
\end{verbatim}

另外,也可以使用字典存储特定信息。因此,人们可以使用描述的缩写和规范形式,得到完整的形式的条目的词尾变化的形式:

\bigskip
\begin{verbatim}
ADN,Acide DésoxyriboNucléique.SIGLE
LADL,Laboratoire d’Automatique Documentaire et Linguistique.SIGLE
SAV,Service Après-Vente.SIGLE
\end{verbatim}



%%%%%%%%%%%%%%%%%%%%%%%%%%%%%%%%%%%%%%%%%%%%%%%%%%%
\section{在字典中查找一个单词}
\index{Dictionnaire!recherche}\index{Dictionnaire!consultation}\index{Consultation d'un dictionnaire}\index{Recherche dans un dictionnaire}
\label{section-dictionary-lookup}
您可以在多个词典用两种方式搜索一个词:

\begin{figure}[h!]
\begin{center}
\includegraphics[width=13cm]{resources/img/fig3-1.png}
\caption{"DELA"菜单}
\end{center}
\end{figure}

\bigskip
\noindent
如果你打开了一本字典,该窗口包含一个字段,允许你执行搜索。如果单词出现在字典中,“查找”按钮,突出显示第一个匹配的条目。如果有多个项目匹配,您可以通过点击这两个按钮状的箭头浏览。
\begin{figure}[h!]
\begin{center}
\includegraphics[width=7cm]{resources/img/fig3-2.png}
\caption{在字典中查找一个单词}
\end{center}
\end{figure}

\bigskip
\noindent
您也可以通过点击“查找”,从“DELA”菜单在多个词典搜索一个单词。然后,你可以选择词典在其中搜索你输入的字。

\begin{figure}[h!]
\begin{center}
\includegraphics[width=7cm]{resources/img/fig3-3.png}
\caption{在多个字典中查找一个单词}
\end{center}
\end{figure}

\bigskip
\noindent

%%%%%%%%%%%%%%%%%%




\section{检查字典格式}
\index{Dictionnaire!vérification} \index{Vérification du format d'un dictionnaire}
当字典很大,就成了乏味的手工检查。 Unitex 包含程序\ \verb+CheckDic+\index{Programmes externes!\verb+CheckDic+}
\index{\verb+CheckDic+} 它会自动检查和\ DELAFDELAS 字典。

\bigskip
\noindent 这个程序实现对输入语法的检查。对于每个格式错误条目,程序将显示行号,该行的内容和错误的性质。分析的结果保存在一个文件名为
\ \verb+CHECK_DIC.TXT+\index{Fichier!\verb+CHECK_DIC.TXT+}中 经验证完后显示出来。除了可能的错误消息,该文件包含所有词形变化形式和规范的形式列表,语法和语义代码列表,并使用词形变化码列表中使用的所有字符的列表。字符列表确保在字典中的字符与在该语言的字母表文件保持一致。每个字符后面的十六进制值。代码清单可以用来验证字典中是否有错别字。
\index{Fichier!alphabet}


\bigskip
\noindent  \verb+CheckDic+ 程序使用未经压缩的字典,也就是为文本文件。该经常被使用的规则是为了提供扩张
\verb+.dic+ \index{Fichier!\verb+.dic+}。 要检查字典的格式,必须
首先通过点击在“DELA”菜单中点击“打开...”。

\begin{figure}[h]
\begin{center}
\includegraphics[width=10cm]{resources/img/fig3-4.png}
\caption{词典例子\label{fig-dictionary-example}}
\end{center}
\end{figure}

\noindent 加载字典图~\ref{fig-dictionary-example}.
要启动自动验证在“DELA”菜单中,点击“检查格式...”。如图\ ~\ref{fig-dictionary-checking} 的窗口显示。
您可在此窗口选择要检查词典的类型。图\ ~\ref{fig-dictionary-example} 的检查结果,显示在图\ ~\ref{fig-dictionary-checking-results} 中。

\bigskip
\noindent 第一个错:没有找到句号。第二个错误:没有找到标注在变化词尾的逗号。第三个错误:该程序没有发现任何语法或语义的代码。




\begin{figure}[!h]
\begin{center}
\includegraphics[width=7cm]{resources/img/fig3-5.png}
\caption{字典的自动验证\label{fig-dictionary-checking}}
\end{center}
\end{figure}

\begin{figure}[!p]
\begin{center}
\includegraphics[height=19.4cm]{resources/img/fig3-6.png}
\caption{字典的自动验证\ \label{fig-dictionary-checking-results}}
\end{center}
\end{figure}


\section{排序}
\index{Dictionnaire!tri}\index{Tri!d'un dictionnaire}

Unitex 处理条目但是不为此排序。但是,如果是为了展示,它往往是最好的字典排序。排序操作有多种标准,首先是文本的语言不同,排序不同。因此泰语的字典是根据字母的不同顺序而排列,所以Unitex为泰语制定了排序方法 (见\ 
 \ref{chap-external-programs}).

\bigskip
\noindent 对于欧洲语言,排序通常是根据字母顺序进行,但有一些不同。事实上,像法国的一些语言考虑某些字符等同。例如,字符之间的差异\ 
\verb+e+ 和\ \texttt{é} 当比较词的时候会被忽略\ \verb+manger+ 和\ 
\texttt{mangés}, 因为之前的字母\ \verb+r+ 和\ \verb+s+ 用于决定顺序。不同的是当前一个字母是相同的时,如我们比较\ \texttt{pêche} 和\ \texttt{pèche}.

\bigskip \index{Alphabet!tri}
\noindent
为了处理这一现象,分拣程序\  \verb+SortTxt+  
\index{\verb+SortTxt+}\index{Programmes externes!\verb+SortTxt+} 使用了定义等价字符的文件。 \index{Équivalence de caractères}  该文件名为
\ \verb+Alphabet_sort.txt+ \index{Fichier!\verb+Alphabet_sort.txt+}  并可以当前用户语言的目录中找到。默认情况下使用文件的首行,以下是法语的例子:


\bigskip
\noindent
\texttt{AÀÂÄaàâä}

\noindent
\texttt{Bb}

\noindent
\texttt{CÇcç}

\noindent
\texttt{Dd}

\noindent
\texttt{EÉÈÊËeéèêë}


\bigskip
\noindent 
在上下文允许的情况下,以上相同行的字符被认为是相等的。
当上下文允许在同一行中的字符被认为是相当的。当你需要比较两个相似的字符时,从左到右进行比较。通过以上我们可看出没有区分大小写和变音符号。


\bigskip
\noindent 如果要排序字典,打开它,在“DELA”菜单中点击“排序辞典”。默认情况下,程序会尝试使用文件\ \verb+Alphabet_sort.txt+。如果这个文件不存在,排序是按照\ Unicode 编码字符的索引完成。通过修改这个文件,你可以定义自己的排序偏好。


\bigskip
\noindent 注:应用字典上的一个文本文件后
\verb+dlf+, \verb+dlc+ 和\ \verb+err+ 有了这个程序会自动排序。
\index{Fichier!\verb+dlf+} \index{Fichier!\verb+dlc+}\index{Fichier!\verb+err+}



\section{全自动词形变化}
\label{section-automatic-inflection}
\index{Flexion automatique}\index{Conjugaison}\index{Déclinaison}\index{Dictionnaire!flexion automatique}
\subsection{单字的词形变化}

就像\ ~\ref{section-DELAS-format} 中定义的一样, DELAS 通常由一个规范的形式和语法或语义编码序列:


\begin{verbatim}
aviatrix,N4+Hum
matrix,N4+Math
radix,N4
\end{verbatim}

\bigskip
\noindent 第一个被读取的码被解释为用于词形变化规范形式的语法名称。有两种可能的形式:

\begin{itemize}
\item \verb+N4+:语法名=\verb+N4.fst2+, 语法码=\verb+N+
	(最长前缀只能是字母)
  \item \verb+N(NC_XXX)+: 语法名=\verb+NC_XXX.fst2+, 语法码=\verb+N+
\end{itemize}

\bigskip
\noindent 这些词形变化文法\index{Grammaires!de flexion}\index{Graphe!de flexion}\index{Transducteur!de flexion} 如有必要,会自动编译。在上面的例子中,所有的词条都会被名为\ \verb+N4+ 的语法进行词形变化。
\bigskip
\noindent 要开始进行词形变化,在“DELA”菜单中点击“影响”。图\ ~\ref{fig-inflection-configuration} 窗口 指出词形变化文法的目录中的词形变化程序。默认情况下,子目录
\verb+Inflection+ 表示当前语言的目录被使用。您也可以指定哪些类型的词典应该被包含。如果遇非法输入,将显示一条错误消息。

\bigskip
\begin{figure}[h]
\begin{center}
\includegraphics[width=8cm]{resources/img/fig3-7.png}
\caption{自动词形变化配置\label{fig-inflection-configuration}}
\end{center}
\end{figure}

\bigskip
\begin{figure}[h]
\begin{center}
\includegraphics[width=4.5cm]{resources/img/fig3-8.png}
\caption{词形变化语法
\texttt{N4}\label{fig-example-inflectional-grammar}}
\end{center}
\end{figure}

\bigskip
\noindent 图\ ~\ref{fig-example-inflectional-grammar} 表示变化语法的一个例子。通过添加或删除后缀从规范化形式转化成词尾变化后的形式,并将词形变化代码(在框中粗体字)添加到字典词条中。

\bigskip
\noindent 在我们的例子中,有两个可行路径。第一个不修改规范形式,并增加了词形变化代码\ \verb+:s+。第二个删除
\verb+L+, 然后加入\ \verb+ces+ 和词形变化码\ \verb+:mp+。

在这里,您可以使用操作器:

\index{\verb+L+}\index{Opérateur!\verb+L+}\index{\verb+R+}\index{Opérateur!\verb+R+}
\index{\verb+C+}\index{Opérateur!\verb+C+}\index{\verb+D+}\index{Opérateur!\verb+D+}
\index{\verb+U+}\index{Opérateur!\verb+U+}
\index{\verb+P+}\index{Opérateur!\verb+P+}
\index{\verb+W+}\index{Opérateur!\verb+W+}
\index{\verb+J+}\index{Opérateur!\verb+J+}
\index{\verb+.+}\index{Opérateur!\verb+.+}
\index{\verb+<R=?>+}\index{Opérateur!\verb+<R=?>+}
\index{\verb+<I=?>+}\index{Opérateur!\verb+<I=?>+}
\index{\verb+<X=n>+}\index{Opérateur!\verb+<X=n>+}

\begin{itemize}
\item \verb+L+ (left) 删除了一项输出;
  	  
\item 从\ \verb+R+ (right) 中恢复信息。在法语中,第一组动词的不定式形式除去词尾的\ \verb+r+ 和第三人称单数进行结合,从末尾起倒数第四个字母\ \texttt{è} 进行改变。\verb+peler+ $\rightarrow$ \texttt{pèle},
  	  \verb+acheter+ $\rightarrow$ \texttt{achète}, \texttt{gérer}
  	  $\rightarrow$ \texttt{gère}, 等等。

  	  
\item \verb+C+ (copy) 通过移动所有其右侧的字母,来重复条目中的字母,
  	  
例如,假设你想自动生成形容词
\ \verb+able+ 如\ \verb+regrettable+ 或\ \verb+réquisitionnable+,
 还有最后的辅音字母名增加了一倍。避免编写一个偏转曲线图对于每个可能的最终辅音,可以使用\ C 操作者重复任何种类的末端辅音;
  
  \item \verb+D+ (delete) 通过移动所有其右侧的字母来删除词条中的字母。例如从词形变化的罗马字\  \verb+european+ 到 \verb+europeni+, 我们用\ \verb+LDRi+,  \verb+L+ 来确定字符\ \verb+a+, \verb+D+ 删除 \verb+a+, 在左边定义 \verb+n+, 随后 \verb+Ri+ 将代替 \verb+n+ 并加入\verb+i+。

\item \verb+U+ 去掉了声调。
	例如 \verb+LLUx+ 应用于
\ \texttt{mangés} 产生了变位\ \verb+mangex+,然后 \ \verb+U+
	将 \ \texttt{é} 替换为\  \verb+e+.

\item \verb+P+ 将首字母大写,例如
	\verb$Px$ 将\ \verb$foo$ 转换为\ \verb$Foox$.
  
\item \verb+W+ 首字母小写。

\item \verb+<R=?>+ 将首字母替换为 \verb+?+.

\item \verb+<I=?>+加入字母 \verb+?+

\item \verb+<X=n>+ 去掉 $n$ 。
\end{itemize}


%\bigskip
\noindent
为使用操作器,考虑 {\it reprendre}~:

\bigskip
\begin{center}
\begin{tabular}{|l|l|l|l|}
\hline
Verbe     & Opérateur & Variable & Résultat\\
\hline
\hline
reprendre & <re> & & reprend\\
reprendre & <\$> & \$ = e & reprendr\\
reprendre & <{\pounds}> &{\pounds}= reprendre & $\varepsilon$ \\
reprendre & <re\$re> & \$ = nd & rep\\
reprendre & <re{\pounds}re> & {\pounds} = prend & \\
reprendre & <\$re> & \$ = d & repren\\
reprendre & <re\$> & \$ =  $\varepsilon$ & reprendre\\
reprendre & <{\pounds}re> & {\pounds} = reprend & $\varepsilon$\\
reprendre & <re{\pounds}> & {\pounds} = prendre & re\\
\hline
\end{tabular}
\end{center}

\bigskip
\noindent
程序\ MultiFlex 允许使用十个变量 \ \$ 它们的名字是\  \$, \$1..., \$9
十个 \ {\pounds} 类型的变量名为\  {\pounds}, {\pounds}1..., {\pounds}9。 此外,多个不同类型的变量可被同一个操作器使用
因此操作器 <{\pounds}3re\$7re> 应用于\ {\it reprendre} 给出\ {\pounds}3 = 
 rep 和 \$7 = \verb+nd+。

\bigskip
\noindent
动词 \verb+accélérer+, \verb+sécher+,第二人称现在时可以由运算\ <é\$er>è\$es~ 产生: 

\begin{center}
\begin{tabular}{lllllllll}
	\verb+accélérer+ & <é\$er> & $\rightarrow$ & accél & \$ = r & + & è\$es &  $\rightarrow$ & \verb+accélères+\\
	\verb+sécher+ & <é\$er> & $\rightarrow$ & s & \$ = ch & + & è\$es & $\rightarrow$ & \verb+sèches+\\
\end{tabular}
\end{center}


\noindent 在这里,获得词形变化的英语形容词 \verb+tranquil+:

\bigskip
\begin{figure}[!ht]
\begin{center}
\includegraphics[width=5cm]{resources/img/fig3-flexion_tranquil.png}
\end{center}
\end{figure}

\noindent 在一些语言中,某些词形变化包含在添加在词根之前的前缀。这是形成过去分词时的情况。同时使用操作器\ \verb+£+ 和\ \verb+$+ 对德语 \ \verb+sprechen+ (parler) 进行词形变化
如图所示现在分词和过去分词\ ~\ref{fig-inflection-sprechen}。

\newpage
\begin{figure}[!htbp]
\begin{center}
\includegraphics[width=5cm]{resources/img/fig3-Advanced_operators_with_Variables-V_sprechen.png}
\caption{词形变化图中的词语,如 {\it sprechen}
\label{fig-inflection-sprechen}}
\end{center}
\end{figure}

\noindent 这里的词形变化德语动词 \verb+sprechen+:

\bigskip
\begin{figure}[!ht]
\begin{center}
\includegraphics[width=5cm]{resources/img/fig3-flexion_sprechen.png}
\end{center}
\end{figure}

\noindent 如果一个人想用动词短语可以使用两种类型的变量\ \$.
图\ ~\ref{fig-inflection-aussprechen} 表示所对应的变量图\ \verb+$1+ et \verb+$2+。

\bigskip
\begin{figure}[!ht]
\begin{center}
\includegraphics[width=10.5cm]{resources/img/fig3-Advanced_operators_with_Variables-V_aussprechen.png}
\caption{词形变化图中的词语,如 {\it aussprechen}
\label{fig-inflection-aussprechen}}
\end{center}
\end{figure}

\noindent 获得德语中的动词 \verb+aussprechen+:
\bigskip
\begin{figure}[!ht]
\begin{center}
\includegraphics[width=5cm]{resources/img/fig3-flexion_aussprechen2.png}
\end{center}
\end{figure}

\bigskip
\noindent \textbf{语义码}
\noindent 在某些语言中,也有实际对应的语义特征,如主动语态标记的\ marketing。这种码和词形变化码相类似,我们记为语义码。为了产生语义码,在一个框的输出的开头插入一个加号。此框应最多只有一个语义码,位于加号之前。如图\ ~\ref{fig-inflection-sem}。

\bigskip
\begin{figure}[!ht]
\begin{center}
\includegraphics[width=6cm]{resources/img/fig3-9sem.png}
\caption{有语义代码的词形变化码\label{fig-inflection-sem}}
\end{center}
\end{figure}

\subsection{复合词的词形变化}
Voir chapitre \ref{chap-multiflex}.

\subsection{闪语族词形变化}
\label{subsection-semitic-inflection}
\index{Langues sémitiques}
像阿拉伯语和伯来语这样的闪语族语言的词形变化无法简单地用以上所描述的运算符来表示。它们的词法遵循另一种逻辑:单词根据辅音结构进行词形变化。

\bigskip
\noindent 首先,我们只编译\ DELAS 字典词根去=区的辅音语法码前的符号\ $ 表明词形变化语法在闪语族模式,ktb  为辅音结构(我们可以在词根区找到)。图\ ~\ref{semitic-grammar} 展示了语法\ \verb+V31-123.grf+ 在语义模式下的运作。词形变化语法使用了阿拉伯写作使用的音译系统。

\bigskip
\noindent \verb+ktb,$V31-123+

\bigskip
\begin{figure}[!ht]
\begin{center}
\includegraphics[width=10cm]{resources/img/fig3-15.png}
\caption{闪语族模式遵循以下规则\ \label{semitic-grammar}}
\end{center}
\end{figure}

\bigskip
\noindent 遵循以下规则:
\begin{enumerate}
\item 可以使用所有词形变化标准运算符() (\verb+L+, \verb+R+, etc.).
\item 一个数字表示词根区的一个字母。(\verb+1+ 第一个
\ \verb+2+ 第二个,等)。在我们的例子中, \verb+1+, \verb+2+ 和\ \verb+3+ 分别代表 \ \verb+k+, \verb+t+ 和\  \verb+b+. 如果你想第九个后指定一个字母,则它必须保留其号码: \verb+<10>+。
\end{enumerate}  

\bigskip
\noindent 以下是由这个语法规则创建的一个\ DELAF 字典条目:\\ 
  
\verb+yakotubu,ktb.V:aI3ms+

\bigskip
\noindent 如果我们只对词根区的辅音进行编码,在两个条目中的辅音相同原因不同的情况下,我们必须在词形变化的语法中对元音进行编码~:\\ 

\verb+Hsb,$V3au	// compter, Hasaba, yaHosubu+

\verb+Hsb,$V3ii	// penser, Hasiba, yaHosibu+

\bigskip
\noindent 
我们可以用<LEMMA>运算符赋值左右的词根区字段。包含这个运算符的方框覆盖了所有词根区,不依赖于字母数量。这个运算符适用于阿拉伯语中阳性形式(阳性形式是由在辅音结构中添加元音获得,阴性形式是添加前缀)。如图\ ~\ref{LEMMA-operator} 在这个例子中,我们同时对词根区的辅音和元音都进行了编码。


\begin{figure}[!ht]
\begin{center}
\includegraphics[width=10cm]{resources/img/fig3-LEMMA-operator.png}
\caption{闪语族模式使用\ <LEMMA>\label{LEMMA-operator} 运算符的语法例子}
\end{center}
\end{figure}

\section{压缩}
\index{Dictionnaire!compression}

Unitex 将压缩字典应用于文本。压缩可以削减字典大小并加速查询。这个操作由\ \verb+Compress+ 程序完成。它可以处理一个文本类型的字典文件(例如\ \verb+mon_dico.dic+)并输出两个文件:\index{Fichier!\verb+.dic+}


\begin{itemize}
  \item \verb+mon_dico.bin+ 包含字典的词形变化形式的最小自动机。
  	  \index{Fichier!\verb+.bin+}
  \item \verb+mon_dico.inf+ \index{Fichier!\verb+.inf+}包含允许通过\ \verb+mon_dico.bin+ 文件中的变化词形重建原始字典的代码。 \verb+mon_dico.bin+.
\end{itemize}

\index{Automate!minimal}
\noindent
\verb+mon_dico.bin+ 中的最小自动机可以表示变化词形,前缀和后缀是选择性添加。例如词语\ \verb+me+, \verb+te+, \verb+se+,
\verb+ma+, \verb+ta+ 和\ \verb+sa+ 的最小自动机。如图\ ~\ref{fig-example-minimal-automaton} 所示。

\bigskip \begin{figure}[!h]
\begin{center}
\includegraphics[width=5cm]{resources/img/fig3-10.png}
\caption{最小自动机的一个例子的表示\label{fig-example-minimal-automaton}}
\end{center}
\end{figure}

\noindent 要压缩字典,打开它,然后单击“压缩到FST”中的“DELA”菜单。压缩是独立于语言和字典的内容。由程序产生的消息显示在不自动关闭一个窗口。您可以看到文件
\verb+.bin+ 的大小, 读取的行数和产生的词形变化的数量。图、 ~\ref{fig-compression-result} 显示了压缩一个单词字典的结果。

\bigskip
\begin{figure}[!h]
\begin{center}
\includegraphics[width=14cm]{resources/img/fig3-11.png}
\caption{压缩的结果\label{fig-compression-result}}
\end{center}
\end{figure}

\bigskip
\noindent 需要指出的是,通常单词字典的压缩比率大约百分之九十五,复合词字典压缩比率大约是百分之五十。

\bigskip
\noindent 注:当闪语族模式在词形变化字典中被广泛使用时,我们可以用一种特殊压缩的算法削减文件\ \verb+.bin+ 和\ \verb+.inf+ 的大小。我们可以在\ préférences globales 中勾选 \ Semitic language 选项或者通过命令启动\ compress 程序,加入选项\ semitic。


\section{词典中的应用}
\label{section-applying-dictionaries}
\index{Dictionnaire!application}

\bigskip
\noindent 
Unitex 可以处理压缩字典(\verb+.bin+) 或图形字典 (\verb+.fst2+)。这些字典用预处理或点击菜单使用。我们现在将具体介绍使用字典的规则。图形字典的使用在\ ~\ref{section-dictionary-graphs} 节介绍。


\subsection{优先级}
\label{section-dictionary-priorities}
\index{Dictionnaire!priorité}\index{Priorité!entre dictionnaires}
优先级规则是:如果文本的单词在字典中被发现,那么这个单词在应用字典中将不会有内部优先权。


\bigskip
\noindent 
这可以减少一些应用字典过程中产生的歧义。例如单词\ \textit{par} 在高尔夫领域可以解释为一个名词。如果我们不想这个用法,我们只需要创建一个包含\ \verb$par,.PREP$ 条目的过滤字典。以此给它一个更高的优先级。这样即使单词字典包含其他条目,也会因为优先级被忽略。
优先级有三个级别。没有后缀且以\ - 结尾的字典优先级最高。以\ + 结尾的字典优先级最低。其他字典优先级一般。应用字典的顺序对优先级没有影响。在命令行中,输入命令:




\bigskip
\noindent
\verb$Dico ex.snt alph.txt ctr+.bin cities-.bin rivers.bin regions-.bin$

\bigskip \noindent 
可以以下列顺序应用字典 (\verb+ex.snt+ 是应用字典的文字, \verb+alph.txt+ 是所使用的字母的文件):




\begin{enumerate}
  \item \verb$cities-.bin$
  \item \verb$regions-.bin$
  \item \verb$rivers.bin$
  \item \verb$ctr+.bin$
\end{enumerate}

\subsection{字典应用规则}
\label{section-transducer-application-rules}

除了优先级规则。应用字典还将考虑大小写和空格。大小写规则如下:

\index{Règles!majuscules et minuscules}

\begin{itemize}
  \item 如果在字典中一个大写字母,那么文本中也将是一个大写;

  \item 如果在字典中的小写的,那么文本中可能为小写或大写。

\end{itemize}

\noindent 
同时条目\ \verb$pierre,.N:fs$ 可以识别\ \verb+pierre+ 和\ \verb+PIERRE+ ,而条目\ \noindent \verb$Pierre,.N+Prénom$ 只能识别\ \verb+pierre+ 和\ \verb+PIERRE+ 。大小写字母由字母文件定义。参数可以在\ \verb+Dico+\index{\verb+Dico+}\index{Programmes externes!\verb+Dico+} \index{Fichier!\verb+Alphabet.txt+}\index{Fichier!alphabet}\index{Alphabet}.\index{Règles!espace} 程序中设置。


\bigskip
\noindent 
考虑空格的规则很简单。如果想要一个字典条目能识别一串文字它们必须有相同的空格。例如,如果字典包含\ \verb+aujourd'hui,.ADV+ ,文本\ \verb+Aujourd' hui+ 不能被识别,因为撇号后有一个空格。




\subsection{字典图}
\label{section-dictionary-graphs}\index{Graphe!dictionnaire}\index{Graphe-dictionnaire}
程序 \verb+Dico+\index{\verb+Dico+}\index{Programmes externes!\verb+Dico+} 同时还能够应用图形的字典。图形默认遵循以下原则:如果我们使用\ \verb+Locate+\index{\verb+Locate+}\index{Programmes externes!\verb+Locate+} 的\ MERGE 模式应用字典,字典将会得出和\ DELAF 每行对应的字符序列\index{DELAF}\index{Dictionnaire!DELAF}。如果我们应用于一个文本,图形字典将\ \index{DELAF}\index{Dictionnaire!DELAF} 的文字标签标记在文字序列上。


\begin{figure}[!p]
\begin{center}
\includegraphics[height=24cm]{resources/img/fig3-12.png}
\caption{识别元素的图形词典\label{elements}}
\end{center}
\end{figure}

\bigskip
\noindent 图表\ref{elements} 为一个可以识别元素符号的字典图。从这个图中我们可以看到压缩字典的首要优势:引号的使用可以强制字典遵循大小写规律。另外,这个字典图能很好的识别\ F 而不是\ FE 。我们惯用的\ DELAF 字典是无法做到这种识别的。
\bigskip
\noindent 图标字典的第二个优势是它们可以扩展之前应用的字典获得的结果,因此,我们可以先应用普通字典,接着我们通过图形字典中\ \verb$NPr+$ ~\ref{graph-NPr} 将大写开头的未知词标注为专有名词。图中\ \verb$+$ 表示较低优先权,因此它将应用于普通字典后。这个图标字典应用于普通字典无法识别的词。中括号对应一个上下文定义。(见\  \ref{section-contexts})。


\begin{figure}[!h]
\begin{center}
\includegraphics[width=10.5cm]{resources/img/fig3-13.png}
\caption{字典图标识由大写字母开头的未知专有名词}
\label{graph-NPr}}
\end{center}
\end{figure}

\bigskip
\noindent 
由于图标字典被\ \verb+Locate+ 程序引擎所应用,它们使用\ \verb+Locate+ 程序许可范围内的所有内容。具体来说,它们可以使用词形过滤器\ \index{Filtre morphologique} (见\ ~\ref{section-filters}) 和词形模式 (见\ ~\ref{section-morphological-mode})。图\ \ref{graph-CR}  所示字典图形使用了过滤器识别罗马数字。需要注意的是,它使用上下文避免一些错误发生,比如\ \verb+C+ 后如果有撇号,它不会被识别为罗马数字。



\begin{figure}[!p]
\begin{center}
\includegraphics[height=24cm]{resources/img/fig3-14.png}
\caption{字典图识别罗马数字\label{graph-CR}}
\end{center}
\end{figure}

\bigskip
\noindent 
图标字典默认使用\ MERGE 模式。它也可以使用\ replace 模式,只需要在名字前加上前缀\ \verb+-r+。也可以加上优先权符号\  \verb-+- 和\ \verb+-+。


\bigskip
\verb?bagpipe-r.fst2  McAdam-r-.fst2  phtirius-r+.fst2?


\subsubsection{导出词形模式字典条目}
\index{Dictionnaire!du mode morphologique}
图形字生成的条目是通过\ \verb+Locate+ 程序完成的,当程序遇到词汇面罩时,我们需要字典帮助。

\bigskip
\noindent然而,这个功能在词汇面罩为词形模式时会受到限制。(见\ ~\ref{section-morphological-mode})我们不能以使用常用的方法混淆图形字典和词形模式。当字典处于词形模式时,词汇面罩需要参考普通词典而不是图形词典。我们有以下几种解决方法:


\begin{itemize}
\item 可以考虑从在形态模式图的一部分调用图的字典。
\item Unitex内部生产的公认形式的字典由图形词典中的文本。如果字典图的名称中包含\ \verb+b+(见命名约定如下图),这本词典是自动生成隐含形态字典模式中,因此,它被咨询的程序时,\verb+Locate+找到符合词法的形态模式。但这种方法只适用于由图的字典词典的初始应用程序识别表格 (见\ ~\ref{section-applying-dictionaries}),而不能应用于那些出现在文本作为标记的部件。
\end{itemize}
如果我们添加 \ \verb+z+ 替代 \ \verb+b+, 文本内部生产字典立即压缩,当其他图形,字典是应用之后,可以协商。
 
\subsubsection{命名约定}
一个图的字典的命名过程如下:\\

\verb$nom(-XYZ)([-+]).fst2$\\

\noindent où:
\begin{itemize}
\item \verb+X+ 取一个值 \ \verb+[rRmM]+: \verb+r+ REPLACE 模式; \verb+M+
MERGE模式(默认模式);
\item \verb+Y+ 取一个值 \ \verb+[bBzZ]+: 选项支配形态字典模式建设(见上文);
\item \verb+Z+ 取一个值\ \verb+[aAlLsS]+: \verb+a+ 意味着该图是在“所有匹配”施加;\verb+l+ 模式"Longest matches" (默认模式); 
\verb+s+ 表示 "Shortest matches".
\end{itemize}


\subsection{形态图词典}
\label{section-morphological-dictionary-graphs}\index{Graphe-dictionnaire!morphologique}
在图形词典中,每个路径必须,默认情况下,产生对包含在字典文本词汇条目。在一个形态字典的图,每个路径必须提供分隔的一个或多个标签的序列由括号和符合\ DELAF 的语法
 (见\ ~\ref{section-DELAF-entry-syntax}).
你的图表的输出将被用作输入来构造文本自动机。我们称之为``形态图形,字典',因为他们的主要目的是提供新的形态在文本自动分析,通过形态学方法
(见 \ \ref{section-morphological-mode}).此功能对于凝集语言如韩语很有用。
要使用图形作为图形的形态字典,我们以斜杠声明它 (/) 作为其释放的第一个字符,如图\ \ref{morphoA}.

\begin{figure}[!ht]
\begin{center}
\includegraphics[width=14cm]{resources/img/fig3-14a.png}
\caption{图的形态字典的例子\label{morphoA}}
\end{center}
\end{figure}

\noindent 规则很简单:任何输出字典图形以斜杠 ( /) 开始
加入文件\ \verb+tags.ind+, \index{\verb+tags.ind+},在索引中定位。
该文件所使用的程序要解释添加到文本自动机。图的语法文字识别
由前缀 \ \verb+un+ 加上形容词构成。 如果应用为图的字典获得在文本自动机的新路径,如图\
\ref{morphoB}> 注意,当在同样,它们之间的链路由虚线显示两个标签匹配分析。

\begin{figure}[!ht]
\begin{center}
\includegraphics[width=15cm]{resources/img/fig3-14b.png}
\caption{由图的形态字典方式加入\ \label{morphoB}}
\end{center}
\end{figure}

\section{参考书目}

该表~\ref{ref-dicos}提供了简单和复合词电子词典的一些参考。有关详细信息,请参阅\ Unitex 网站上的参考页: \url{http://www-igm.univ-mlv.fr/~unitex}

\bigskip
\begin{table}[!h]
\begin{center}
\begin{tabular}{|l|c|c|}
\hline
\textbf{Langue} & \textbf{Mots simples} & \textbf{Mots composés} \\
\hline
English & \cite{klarsfeld}, \cite{monceaux-1995} & \cite{delac-anglais},
\cite{these-Savary} \\
\hline
French & \cite{formes-ambigues}, \cite{dicos-francais}, \cite{jacques-1995} & \cite{dicos-francais},
\cite{Gross96},
\cite{max-1993},
\cite{syntaxe-de-ladverbe} \\
\hline
Modern Greek & \cite{modern-greek}, \cite{matthieu-anastasia}, \cite{these-tita} & \cite{tita-2002},
\cite{anastasia-2002} \\
\hline
Italian & \cite{delaf-italien}, \cite{delaf-italien-book} & \cite{composes-italien} \\
\hline
Spanish & \cite{blanco-2000} & \cite{blanco-1997} \\
\hline
Portuguese & \cite{eleuterio1995}, \cite{ranchhod1996b}, \cite{ranchhodd1998},
\cite{muniz2005} & \cite{ranchhod1991}, \cite{ranchhodd1998} \\
\hline
\end{tabular}
\caption{对电子词典的一些引用\label{ref-dicos}}
\end{center}
\end{table}
