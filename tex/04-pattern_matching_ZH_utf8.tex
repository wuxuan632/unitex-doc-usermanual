\chapter{正则表达式搜索}
\label{chap-regexp}

本章介绍了如何使用正则表达式来搜索文本的简单模式。

\section{定义}
\index{Expression rationnelle}\index{Expression régulière}

本章的目的不是对正式语言的介绍,而是展示了如何在\ Unitex 中使用正则表达式,来查找简单的模式,感兴趣的读者可以参考更多相关作品。


\bigskip \noindent 一个正则表达式,又称正规表示法,可以表示为:

\begin{itemize}
  \item 一个\ \verb+token+(\ \verb+livre+ )或者一个词汇掩模(\ \verb+<manger.V>+ );
  \item 在文本中的特定位置:开始\ \verb+{^}+ 或结尾\ \verb+{$}+;
  \item 两个正则表达式的连接(\ \verb+je mange+ );\index{Concaténation d'expressions rationnelles}
  \item 两个正则表达式的结合(\ \verb$Pierre+Paul$ );\index{Union d'expressions rationnelles} 
  \item Kleene星号的正则表达式 (\ \verb+très*+ )。\index{Étoile de Kleene}
\end{itemize}


\section{Token}
\index{Unité lexicale}

在一个正则表达式,\ \verb+token+ 的定义在\ \ref{tokenization}(页码 \pageref{tokenization})。说明了句号,加号,星号,小于号,开闭括号,和双引号有特殊意义,因此,有必要在它们前面加上一个转义字符\ \verb+\+ 如果您要寻找他们。下面是\ \verb+token+ 正确的一些例子:\index{\verb+\+}

\begin{verbatim}
chat
\.
<N:ms>
{S}
\end{verbatim}

\index{Respect!de la casse}\index{Respect!des minuscules/majuscules}
\noindent  默认情况下,\ Unitex 被设置为让小写模式也能找到大写并与其匹配。可以使用引号来强制区分大小写。因此,\ \verb+"pierre"+ 只能识别\ \verb+pierre+ 而不能识别\ \verb+Pierre+ 和\ \verb+PIERRE+。

\bigskip
\noindent 注意:为了使空格强制性的,它需要用引号括起来。


\index{Espace!obligatoire}


\section{词汇掩模}
\index{Masque lexical} \ \ 词汇掩模是\ \verb+token+ 或一串\ \verb+token+ 对应的搜索查询。

\subsection{特殊符号}
\label{section-special-symbols}
\index{Méta-symboles}

有两种词汇掩模。第一种在章节介绍~\ref{section-sentence-splitting},包含除了\ \verb$<PNC>$  和\ \verb+<^>+。(符号\ \verb$<PNC>$ 的特殊符号和元符号,能够匹配标点符号,只有在预处理阶段有效~; \verb+<^>+ 匹配换行,但当所有的换行都被空格替换,此符号在搜索词汇掩模时就不适用了。) 可用于搜索文本的模式的元符号的情况如下~:

\index{\verb+<MOT>+}\index{\verb+<MIN>+}\index{\verb+<MAJ>+}\index{\verb+<PRE>+}\index{\verb+<NB>+}
\index{\verb+#+}\index{\verb+<E>+}\index{\verb+<DIC>+}\index{\verb+<SDIC>+}\index{\verb+<CDIC>+}
\index{\verb+<TDIC>+}\index{\verb+<WORD>+}\index{\verb+<UPPER>+}\index{\verb+<LOWER>+}\index{\verb+<FIRST>+}
\begin{itemize}
  \item \verb+<E>+ : 空词, 或者ε。识别出空序列;
  \item \verb+<TOKEN>+ : 识别任意\ \verb+token+ 除了默认用于形态滤波器的空格;
  \item \verb+<WORD>+ : 识别任意字母组成的\ \verb+token+;
  \item \verb+<LOWER>+ : 识别任意小写字母组成的\ \verb+token+;
  \item \verb+<UPPER>+ : 识别任意大写字母组成的\ \verb+token+;
  \item \verb+<FIRST>+ : 识别任意首字母大写组成的\ \verb+token+;
  \item \verb+<DIC>+ : 识别任意出现在文本字典的词;
  \item \verb+<SDIC>+ : 识别任意出现在文本字典的单词;  \index{Mots!simples}
  \item \verb+<CDIC>+ : 识别任意出现在文本字典的复合词;
  	  \index{Mots!composés}
  \item \verb+<TDIC>+ : 识别任何标签\ \verb+token+ 比如\ \verb+{XXX,XXX.XXX}+;
  \item \verb+<NB>+ : 识别任何连续的数字串(\ 1234 可以被识别,但是\ 1 234 不可以);
  \item \verb+#+ :禁止空格的存在。\index{Espace!interdit}
\end{itemize}

\noindent  对应旧代码\ \verb+<WORD>+,\ \verb+<LOWER>+,\ \verb+<UPPER>+ 和\ \verb+<FIRST>+ 分别是\ \verb+<MOT>+,\ \verb+<MIN>+,\ \verb+<MAJ>+ 和\ \verb+<PRE>+。它们仍然可以使用,以保持现有图形的系统兼容性,但是现在它们是过时的,也就是说为了最新版本的运行,我们建议在图形设计中避免它们的使用\footnote{从3.1版本开始,修改于2015年十月二号。},以避免不必要词汇掩模的使用增加。

\bigskip
\noindent 注意:如章节\ref{tokenization}所述,任何\ métas 都不能用来识别标记\ \verb+{STOP}+\index{\verb+{STOP}+},\ \verb+<TOKEN>+ 也不能。

\subsection{参考字典提供的信息}
\index{Masque lexical}\index{Dictionnaire!référence aux informations du}
\index{Référence aux informations dans les dictionnaires}\index{Dictionnaire!du texte}

第二种词汇掩模包括来那些能够查找文本字典的信息.四种可能的形式是~:
\bigskip
\begin{itemize}
\item \verb+<lire>+: 识别所有包含\ \verb+lire+ 的词条作为标准形式。如果\  \verb+lire+ 是一个语法代码同时又是一个语义代码,我们说这种形式是歧义的;
  \item \verb+<lire.>+:识别所有包含\ \verb+lire+的词条作为标准形式。 这个词汇掩模在上一种情况中不是歧义的;
  \item \verb+<be.V>+: 识别所有包含\ \verb+lire+ 和语法代码\ \verb+V+ 的词条,作为标准形式;
  \item \verb+<V>+: 识别所有包含语法代码\ \verb+V+ 的词条。 这个词汇掩模是歧义的如同第一种情况。为了消除歧义性,我们可以使用\ \verb+<.V>+ 或者\ \verb$<+V>$;
   \index{Étiquette lexicale}
\item \verb+{lirons,lire.V}+ 或\ \verb+<lirons,lire.V>+ : 识别所有包含\ \verb+lirons+ 的词条作为词形变化形式,包含\ \verb+lire+ 的词条作为标准形式和包含语法代码\ \verb+V+。 如果您在文本自动机进行了说明词的多义性工作,这种类型的词汇掩模是唯一有用的。\index{Texte!automate du}\index{Automate!du texte} 当我们搜索一篇文章时,这个掩模和\ \verb+token+ \ \verb+lirons+ 识别相同的东西。
\end{itemize}

\subsection{语法和语义约束}

 前面的词汇掩模的例子是简单的。为了表达更为复杂的模式,在语法,语义代码中间用\ \verb$+$ 分隔开。如果有很多代码,符号\ \verb$+$ 可以解释成`'和''~: 在字典中一个词条只有当它所有的代码都包括在掩模中,才能被识别。
  掩模\ \verb$<N+z1>$ 识别这些词条:

\bigskip
\noindent
\texttt{broderies,broderie.N+z1:fp}

\noindent
\texttt{capitales europ\'eennes,capitale europ\'eenne.N+NA+Conc+HumColl+z1:fp}

\bigskip
\noindent 不识别以下:

\bigskip
\noindent
\texttt{Descartes,Ren\'e Descartes.N+Hum+NPropre:ms}

\noindent
\texttt{habitu\'e,.A+z1:ms}

\bigskip
\noindent  我们可以排除代码,在它之前加上字符\ \verb+~+ 而不用\ \verb$+$。\index{Exclusion des codes grammaticaux et sémantiques}\index{\verb+~+}
\index{Négation!d’une propriété}
为了能被识别,一个词条必须包括所有掩模所要求的代码,不包含任何它禁止的代码。 举个例子,\ \verb$<A~z3>$ 识别所有包含代码\ \verb+A+,而不包含代码\ \verb+z3+ 的词条(cf.table~\ref{tab-semantic-codes})
\footnote{如果字典写入一个由\ \texttt{A+z3} 和\ \texttt{A} 入口开头的词条,该词条由\ \texttt{<A+z3>} 和\ \texttt{A} 识别。}。
如果我们想要搜索一个包含字符\ \verb$~$ 的代码,我们需要在它前面加一个字符\ \verb+\+。

\bigskip
\noindent 注:2.1版本之前,否定运算符是减号。如果想要使用旧图不加修改,需要在命令行加\ \verb+Locate+ 和操作\ \verb+-g minus+。

\bigskip
\noindent  词汇掩模的句法在语法码(table~\ref{tab-grammatical-codes})和语义码(table~\ref{tab-semantic-codes})之间没有差别。在\ DELAF 电子词典里,语法码是那些出现在第一个和编码语法范畴,但在\ Unitex 的词汇掩模中,语法码和语义码的出现顺序不重要。以下三个词汇掩模是等价的:

\begin{verbatim}
<N~Hum+z1>
<z1+N~Hum>
<~Hum+z1+N>
\end{verbatim}

\noindent 一个词汇掩模可以包含语义码而没有语法范畴的代码。

\bigskip
\noindent 注意:使用只有禁止码的掩模是不可以的。\ \verb+<~N>+ 和\ \verb+<~A~z1>+ 是不正确的掩模。但是,您可以运用上下文解释这个约束。(见章节~\ref{section-contexts})


\subsection{变位限制}
\index{Contraintes flexionnelles}
另外,也可以指定有关变位代码约束。这些限制都必须先通过至少一个语法或语义代码。他们遵循相同格式的约束,由字典中的变位代码组成。以下是一些使用变位限制的词汇掩模的例子:


\begin{itemize}
  \item \verb+<A:m>+ 识别一个阳性的形容词~;
  \item \verb+<A:mp>+ 识别一个阳性复数形容词。
\end{itemize}

\noindent 变位代码由字母引入\ \verb+:+ 有一个或多个字母构成,而且每个字母传达一个信息。先从一个单独变位代码组成的词汇条目和掩模的简单情况开始。为了词条条目\ $E$ 被掩模\ $M$ 识别,需要\ $E$ 的变位代码中包含了\ $M$ 的变位代码中的所有字符~:

\bigskip
$E$=\verb$sépare,séparer.V:Y2s$

$M$=\verb$<V:Y2>$

\bigskip
\noindent $E$的\ \verb+Y2s+ 包括字符\ \verb+Y+ 和\ \verb+2+。$E$ 至少包括一个\ \verb+Y2+, 词汇掩模\ $M$ 识别条目\ $E$。

\bigskip
\noindent 变位代码内部字符的顺序是无关紧要的。所有的语法和语义代码必须先于变位代码。 

\bigskip
\noindent 如果几个变位码存在于一个词法掩模,符号\ \verb+:+ 表示\ `' 或\ ''~:

\begin{itemize}
  \item \verb+<A:mp:f>+ 同时匹配\ \verb+<A:mp>+ 和\ \verb+<A:f>+~; 它识别要么阳性复数形容词,要么阴性形容词~;
  \item \verb+<V:2:3>+ 识别第2人称或第3人称;排除了既没有第二或第三人称(不定式,过去分词和现在分词)的所有时态以及以第一人称变位的时态。 
\end{itemize}

\noindent 为了一个字典词条\ $E$ 被掩模\ $M$ 识别,需要\ $E$ 中的至少一个变位代码包含\ $M$ 的至少一种变位代码的所有字符。考虑以下的例子:

\bigskip
$E$=\verb$sépare,séparer.V:W:P1s:P3s:S1s:S3s:Y2s$

$M$=\verb$<V:P2s:Y2>$

\bigskip
\noindent  没有同时包含\ \verb+P+,\ \verb+2+ 和\ \verb+s+ 的变位代码\ $E$。然而,$E$ 代码\ \verb+Y2s+ 却包含字符\ \verb+Y+ 和\ \verb+2+。代码\ \verb+Y2+ 至少包括一个代码\ $E$,因此词汇掩模\ $M$能识别条目\ $E$。


\subsection{词汇掩模的否定}
\index{Négation!d’un masque lexical}
\index{\verb+"!+}
可以通过排列字符~\ \verb+!+ 于字符~\ \verb+<+ 后面来否定词汇掩模。否定是可能的,当掩模为 \ \verb+<WORD>+,\ \verb+<LOWER>+,\ \verb+<UPPER>+,
\ \verb+<FIRST>+ \footnote{和其等同减值\ <MOT>,\ <MIN>,\  <MAJ>,\ <PRE>。 见章节~\ref{section-special-symbols}。},
\verb+<DIC>+  以及只包含语法,语义和变位的词汇掩模(\textit{i.e.} \verb$<!V~z3:P3>$)。掩模\ \verb+#+ 和 \ \verb+""+ 彼此是否定的。  
\index{\verb+<E>+}\index{\verb+<NB>+}\index{\verb+#+}
掩模\ \verb$<!WORD>$ 能够识别所有的不由字母组成的\ \verb+token+,除了句子分隔符\ \verb+{S}+ 和标记 \ \verb+{STOP}+。
否定对 \ \verb+<NB>+, \ \verb+<SDIC>+,\ \verb+<CDIC>+,\ \verb+<TDIC>+ 和\ \verb+<TOKEN>+ 没有影响。

\bigskip
\noindent  该否定以一种特别方式解释,当掩模是 
\ \verb+<!DIC>+,\ \verb+<!LOWER>+,\ \verb+<!UPPER>+ 和
\ \verb+<!FIRST>+ \footnote{和其等同减值\ <MIN>,\ <MAJ> 和\ <PRE> 时。 参考章节~\ref{section-special-symbols}。}。
\index{\verb+<DIC>+}\index{\verb+<LOWER>+}\index{\verb+<UPPER>+}\index{\verb+<FIRST>+}
\index{\verb+<MIN>+}\index{\verb+<MAJ>+}\index{\verb+<PRE>+}
 这些掩模只识别字母序列的形式,而不是识别不能被非否定掩模匹配的形式。 因此, 掩模 \ \verb+<!DIC>+ 让您找到未知词语的文本
 (cf. figure~\ref{fig-search-<!DIC>})。这些未知的形式大多是专有名词,新词和拼写错误。

\bigskip
\begin{figure}[h]
\begin{center}
\includegraphics[width=15cm]{resources/img/fig4-1.png}
\caption{\ méta 查找的结果 \texttt{<!DIC>}\label{fig-search-<!DIC>}}
\end{center}
\end{figure}

\bigskip
\noindent 词汇掩模的否定,例如 \ \verb+<V:G>+ 能匹配所有词,除了能被该掩模匹配的词。 然而,掩模 \ \verb+<!V:G>+ 无法识别英文形式\ \emph{being},即使存在于词典同名非动词条目:


\begin{verbatim}
being,.A
being,.N+Abst:s
being,.N+Hum:s
\end{verbatim}
\index{Mots!inconnus}

\noindent 下面是不同约束类型的词汇掩模例子:

\begin{itemize}
  \item \verb$<A~Hum:fs>$ : 非人类的单数阴性形容词;
  \item \verb+<lire.V:P:F>+ : 动词 \ \textit{lire} 的现在时态或将来时态;
  \item \verb$<suis,suivre.V>$ : 变位动词 \ \textit{suivre} 的配合词\ \textit{suis} 
  	  (动词原形 \textit{être});
  \item \verb$<facteur.N~Hum>$ : 所有包含标准形式\ \textit{facteur} 的名词入口,且无语义代码\ \verb+Hum+;
  \item \verb$<!ADV>$ : 所有不是副词的词;
  \item \verb$<!WORD>$ : 所有不是字母组成的\ \verb+token+,除了分隔符(见章节~\ref{fig-search-<!WORD>})。 该掩模不识别句子分隔符\ \verb+{S}+ 也不识别标签\ \verb+{STOP}+。
  	  \index{\verb+{S}+}\index{Séparateur!de phrases}\index{\verb+{STOP}+}
\end{itemize}

\bigskip
\begin{figure}[h]
\begin{center}
\includegraphics[width=15cm]{resources/img/fig4-2.png}
\caption{\ méta 查找的结果
\texttt{<!WORD>}\label{fig-search-<!WORD>}}
\end{center}
\end{figure}

\section{级联}
\index{Concaténation d'expressions rationnelles}\index{\verb+.+}\index{Opérateur!concaténation}

有三种连接正则表达式的方法。第一种使用由点表示的级联运算符。表达式如下:

\begin{verbatim}
<DET>.<N>
\end{verbatim}

\noindent 识别一个由名词跟着的限定词。该空格也可以用于级联,以及空字符串。
以下为表达式例子: 


\begin{verbatim}
le <A> chat
le<A>chat
\end{verbatim}

\noindent 识别\ \verb+token+\ \textit{le},后面跟着的是形容词和\ \verb+token+\ \textit{chat}。括号\index{Parenthèses}被用作正则表达式的分隔符。
以下表达式都是等效的:


\begin{verbatim}
le <A> chat
(le <A>)chat
le.<A>chat
(le).<A> chat
(le.(<A>)) (chat)
\end{verbatim}

\section{合并}
\index{Union d'expressions rationnelles}\index{\verb$+$}
\index{Opérateur!disjonction}
正则表达式的合并通过字符\ \verb$+$ 分隔开。 
表达式如下:

\begin{verbatim}
(je+tu+il+elle+on+nous+vous+ils+elles) <V>
\end{verbatim}

\noindent
识别代词后跟一个动词。如果在表达式中的元素是可选的,它足以使用该元素和空字符的联合。 
\index{\verb+<E>+} 例子:

\bigskip
\noindent \verb$le(petit+<E>)chat$ 识别序列\ \textit{le chat} 和\ \textit{le petit chat}。

\smallskip
\noindent \verb$(<E>+franco-)(anglais+belge)$ 识别\ \textit{anglais},\ \textit{belge},
\ \textit{franco-anglais} 和\ \textit{franco-belge}。

\section{Kleene星号}
\index{Étoile de Kleene}\index{\verb+*+}\index{Opérateur!étoile de Kleene}\index{Opérateur!itération}
Kleene 星号,由符号 \ \verb+*+ 表示,可以识别零,出现一个或多个在表达式中。星号应该位于相关元素的右边。
表达式如下:


\begin{verbatim}
il fait très* froid
\end{verbatim}

\noindent 识别\ \textit{il fait froid}, \ \textit{il fait très froid},
\ \textit{il fait très très froid}, 等等。 星号较其他运算符有较
高优先级。为了在复杂的表达式中使用星号,需要使用括号。 
表达式如下:


\begin{verbatim}
0,(0+1+2+3+4+5+6+7+8+9)*
\end{verbatim}

\noindent 识别零,后跟一个逗号和一串空数字。

\bigskip
\noindent 注意 : 禁止用正则表达式搜索空词。如果我们尝试查找 
\verb$(0+1+2+3+4+5+6+7+8+9)*$, 系统将报错,
如图~\ref{fig-epsilon-error}。


\bigskip
\begin{figure}[h]
\begin{center}
\includegraphics[width=14cm]{resources/img/fig4-3.png}
\caption{当搜索识别到空词的表达式时,出现错误 \label{fig-epsilon-error}}
\end{center}
\end{figure}


\section{形态滤波器}
\label{section-filters}
\index{Filtre morphologique}

将形态滤波器用于词汇单位的查找是可能的。为此, 有必要立即跟随
由在双括号的滤波器中找到的词汇单位:


\bigskip
\noindent
\textit{motif}\verb$<<$\textit{motif morphologique}\verb$>>$ \\


\bigskip\index{Expression régulière}\index{Expression rationnelle}\index{POSIX}
\noindent 该形态滤波器表示为\ POSIX 格式的正则表达式(见 \cite{TRE} 详细语法)。 下面是基本过滤器的一些例子:




\begin{itemize}
  \item \verb$<<ss>>$: 包含 \verb$ss$。
  \item \verb$<<^a>>$: 开始于 \verb$a$。
  \item \verb+<<ez$>>+: 以 \verb$ez$结束。
  \item \verb$<<a.s>>$: 包含 \verb$a$ 后跟任何一个字符,  后跟 \verb$s$。
  \item \verb$<<a.*s>>$: 包含 \verb$a$ 后跟任何多个字符, 后跟\verb$s$。
  \item \verb$<<ss|tt>>$: 包含 \verb$ss$ 或者 \verb$tt$。
  \item \verb$<<[aeiouy]>>$: 包含无重音符号原音。
  \item \verb$<<[aeiouy]{3,5}>>$: 包含一串无重音符号原音, 其长度在3到5之间。
  \item \verb$<<es?>>$: 包含 \verb$e$ 后可跟一个字符 \verb$s$。
  \item \verb$<<ss[^e]?>>$: 包含 \verb$ss$ 后跟非原音字符 \verb$e$。
\end{itemize}

\bigskip
\noindent 可以组合这些基本过滤器,以形成更复杂的过滤器:

\begin{itemize}
\item \verb+<<[ai]ble$>>+: 结束于 \ \verb$able$ 或者\ \verb$ible$。
\item \verb$<<^(anti|pro)-?>>$: 开始于\ \verb$anti$ 或者\ \verb$pro$,后可跟一个破折号。
  \item \verb+<<^([rst][aeiouy]){2,}$>>+: 由两个或更多词组成的,由 \ \verb$r$,\ \verb$s$ 或者\ \verb$t$ 开头,后跟一个无重音符号原音。
  \item \verb!<<^([^l]|l[^e])>>!: 不由\ \verb$l$ 开头或者第二个词不是\ \verb$e$,也就是说无论什么词除了\ \verb$le$ 开头的。这种限制更好的说明在章节(见~\ref{section-contexts})。
\end{itemize}

\noindent 一般情况下,单独一个形态滤波器被认为是将其应用于词汇掩模\ \verb$<TOKEN>$, 这里指除了空格和\ \verb+{STOP}+的任意\ \verb+token+。
另一方面,当过滤器紧跟一个词汇掩模,它适用于由词法掩模识别的。这里有这样的组合的一些例子:



\begin{itemize}
  \item \verb+<V:K><<i$>>+: 过去分词由 \ \verb$i$ 结尾。
  \item \verb!<CDIC><<->>!: 复合词中包含破折号。
  \item \verb!<CDIC><< .* >>!: 含有至少两个空格的复合词。
  \item \verb!<A:fs><<^pro>>!:由\verb$pro$开始的阴性单数形容词。
  \item \verb!<DET><<^([^u]|(u[^n])|(un.+))>>!: 不同于\ \verb$un$ 的限定词。
  \item \verb+<!DIC><<es$>>+: 不在字典里的词而且结束于\ \verb$es$。
  \item \verb!<V:S:T><<uiss>>!: 包含\ \verb$uiss$ 的现在、过去虚拟式动词。
\end{itemize}

\noindent \index{Respect!de la casse}\index{Respect!des minuscules/majuscules}标记 : 一般情况下,默认情况下,形态滤波器受词汇掩模一样的变化。 因此,过滤器 \ \verb$<<^é>>$ 能识别所有\ \texttt{é},\ \texttt{E} 或 \ \texttt{É} 开头的单词。为了增
强严格遵守过滤器的容量,需要在过滤后立即加上\ \verb+_f_+。 例子: \verb+<A><<^é>>_f_+。



\section{搜索}
\index{Configuration de la recherche}
\subsection{搜索配置}
\label{section-configuration-recherche}
为了搜索一个表达式, 首先需要打开一篇文章 (见章节~\ref{chap-text})。 然后点击 "  Locate Pattern..."在菜单" Text "中。 窗口如图片所示~\ref{fig-regexp-search-configuration} 

\bigskip
\begin{figure}[h]
\begin{center}
\includegraphics[width=8.8cm]{resources/img/fig4-4.png}
\caption{表达式搜索的窗口\label{fig-regexp-search-configuration}}
\end{center}
\end{figure}

\noindent "Locate Pattern"框让您可以选择正则表达式或者一个语法。 点击 "Regular expression"。


\bigskip
\noindent  "Index"框可以选择识别模式: 

\bigskip
\index{Shortest matches}\index{Longest matches}\index{All matches}
\index{Occurrences!les plus courtes}\index{Occurrences!les plus longues}\index{Occurrences!toutes}
\begin{itemize}
  \item "Shortest matches" : 优先考虑最短序列。
  	  比如, 如果程序识别这两个序列 \ \textit{very hot chili} 和\ \textit{very hot}, 第一个会被丢弃;
  \item "Longest matches" : 优先考虑最长序列。 这是默认模式;
  \item "All matches" : 考虑所有识别出的序列。
\end{itemize}

\bigskip
\noindent  "Search limitation"框用于限制一个特定的情况数量。默认情况,搜索情况数量是\  200。
\index{Occurrences!nombre}

\bigskip
\noindent  "Grammar outputs"的选项和正则表达式无关。具体情况描述于章节 
~\ref{section-applying-graphs-to-text}。 同样为了选项标签
"Advanced options"(见章节\ref{section-advanced-search-options})。

\bigskip
\noindent 在 "Search algorithm"框, 我们规定如果我们想对文章用程序\ \verb+Locate+ 进行搜索,或者在自动程序 \ \verb+LocateTfst+。默认情况下使用程序\ \verb+Locate+。如果您想使用\ \verb+LocateTfst+, 可阅读章节 \ref{section-locate-tfst}。

\bigskip
\noindent 输入表达式,然后单击"Search"以开始搜索。Unitex 将表达式转变成了一种\ \verb+.grf+\index{Fichier!\verb+.grf+} 格式的语法。此语法将被编译成格式\ \verb+.fst2+\index{Fichier!\verb+.fst2+} 的语法,这将用于程序搜索。


\subsection{显示结果}
\label{section-display-occurrences}
一旦搜索结束, 窗口如图~\ref{fig-search-results}出现, 指出找到的匹配个数,识别的\ \verb+token+ 的总数,和它占文章\ \verb+token+ 总数的比例。


\bigskip
\begin{figure}[h]
\begin{center}
\includegraphics[width=6.5cm]{resources/img/fig4-5.png}
\caption{搜索结果 \label{fig-search-results}}
\end{center}
\end{figure}

\noindent 点击 "OK"后, 您将会看到窗口如图
\ref{fig-configuration-concordance} 它显示匹配的事件。您也可以通过点击"Display Located Sequences..." 在菜单 "Text"来显示该窗口。
我们把 \ \textit{concordance}\index{Concordance} 叫做事件清单。


\bigskip
\begin{figure}[h]
\begin{center}
\includegraphics[width=11cm]{resources/img/fig4-6.png}
\caption{找到发生事件的显示配置\label{fig-configuration-concordance}}
\end{center}
\end{figure}

\bigskip
\noindent "Modify text"让我们有可能把找到的事件替换成最终输
出。这在章节~\ref{chap-advanced-grammars}提到。

\bigskip
\noindent  "Extract units"框让您可以用所有包含或不包含
匹配单元的句子来创建一个文本文件。按钮 "Set File"让您选择
输出文件。然后点击 "Extract matching units" 或 "Extract unmatching units" 取决于您是否喜欢句子包含匹配单元或否。


\bigskip
\noindent 在 "Show Matching Sequences in Context"框,您可以选择匹配单元显示的左右边文章字符长度。如果匹配具有比文章右部更少的字符,这行将会以必要字符显示。如果匹配具有比文章右部更多的字符,它会被完全显示出来。 


\bigskip
\noindent 注:在泰国,上下文的大小是由可显示的字符衡量的,而不是实际的字符,这有利于保持匹配单元行的直线性,尽管连接其他字母的区别符号不是像一般字符那样显示的。 


\index{Tri!de concordance}
\index{Contexte!concordance}
\bigskip
\noindent 您可以在"Sort According to"中选择排序顺序。"Text Order" 模式按事件在文章出现的顺序显示。其他六种模式允许列的排列。行的三个区域分别是左侧文本,匹配事件,右侧文本。匹配事件和右侧文本从左向右排序。左侧文本从右向左排序。使用的默认模式是 "Center, Left Col."。索引生成 \ HTML\index{Fichier!HTML}
格式的文件。

\bigskip
\noindent 如果一个索引对应上千个事件,它最好用一个浏览器来显示  ( \ Firefox \cite{Firefox}, Netscape \cite{Netscape}, 
Internet Explorer, 等等)。\index{Navigateur web}
\newline
然后选择 "Use a web browser to view the concordance" (参见图
	~\ref{fig-configuration-concordance})。
	当事件超过3000,该选项将默认自动执行。为了自定义使用的浏览器, 点击 "Preferences..." 在菜单 "Info"。单击选项卡 "Text Presentation" 然后在"Html Viewer" 选择使用的程序。
 (参见图~\ref{fig-browser-selection})。

\bigskip
\noindent \index{Cadre des concordances} 如果您选择在 \ Unitex 内部打开索引,\ref{fig-example-concordance}。 
选项 "Enable links"默认运行,以保证匹配事件的超链接。 而且, 当我们点击一个事件,文本窗口被打开,相应的序列被突出显示。此外,如果文本可自动建立,如果此窗口未图标化,包含索引的句子自动器将被装载。如果我们选择选项"Allow concordance edition", 我们不能点击索引, 但是我们可以作为文本修改它。可以用光标移
动它,如果我们要处理大量的文章,使用索引会变得很方便。


\begin{figure}[h]
\begin{center}
\includegraphics[width=8cm]{resources/img/fig4-7.png}
\caption{选择一个浏览器来显示词汇索引\label{fig-browser-selection}}
\end{center}
\end{figure}

\begin{figure}[!p]
\begin{center}
\includegraphics[height=18cm]{resources/img/fig4-8.png}
\caption{一致性的例子\label{fig-example-concordance}}
\end{center}
\end{figure}

\clearpage
\subsection{统计}
\label{section-statistics}
如果我们在``Located sequences..''选择 ``Statistics'' ,
显示如图的面板~\ref{fig-statistics}。该面板可以让您从之前的索引序列中得到一些统计数据。 

\bigskip
\begin{figure}[!h]
\begin{center}
\includegraphics[width=11cm]{resources/img/fig4-9.png}
\caption{统计面板 \label{fig-statistics}}
\end{center}
\end{figure}

\bigskip
\noindent 在面板 ``Mode'',您可以选择您想要的统计方式: 
\begin{itemize}
  \item 搭配词由频率: 指出文章中的\ \verb+token+ 在匹配文章,
  \item 搭配词由Z值: 同上,  (在匹配文章和整个文库的匹配
  词的数量, 搭配词的Z值)
  \item 上下文由频率: 指出在左右侧文本\ \verb+token+  (见下文)。 ``count'' 是识别序列的匹配的总数。
\end{itemize}

\bigskip
\noindent 在第二个面板, 我们选择左右侧文本的长度为了使用无空格\ \verb+token+。
注意: 这个上下文的概念和语法的不同。


\bigskip
\noindent 在最后一个面板, 我们可以允许或不允许大小写转换。
在允许的情况下, \verb$the$ 和  \ \verb$THE$ 是同样的\ \verb+token+,计数总和是  \ \verb$the$ 的总数加上 \ \verb$THE$ 的总数。

\bigskip
\noindent 下图显示了在计算各模式的查询统计
 \ \verb$<have>$ 于  \ \verb$ivanhoe.snt$.


\bigskip
\begin{figure}[!h]
\begin{center}
\includegraphics[width=11cm]{resources/img/fig4-10.png}
\caption{事件的背景下+左+右+上下文匹配数\label{fig-statistics-mode0}}
\end{center}
\end{figure}

\begin{figure}[!h]
\begin{center}
\includegraphics[width=11cm]{resources/img/fig4-11.png}
\caption{搭配组合\label{fig-statistics-mode1}}
\end{center}
\end{figure}

\begin{figure}[!h]
\begin{center}
\includegraphics[width=12cm]{resources/img/fig4-12.png}
\caption{搭配组合,统计和其他信息 \label{fig-statistics-mode2}}
\end{center}
\end{figure}
